\chapter{Cryptographic Protocols}
\label{ch:protocols}
	\section{Zero-Knowledge Proofs}

Read \S 14.1 of \cite{tw}.

\begin{problem}[10 points]
	\S 14.3 \#1 of \cite {tw}
\end{problem}

\begin{problem}[15 points]
	\S 14.3 \#2[a,b] of \cite {tw}
\end{problem}

\begin{problem}[10 points]
	\S 14.3 \#2[c] of \cite {tw}
\end{problem}

\begin{problem}[10 points]
	\S 14.3 \#3[a] of \cite {tw}
\end{problem}

\begin{problem}[15 points]
	\S 14.3 \#3[b, c] of \cite {tw}
\end{problem}

\begin{problem}[15 points]
	\S 14.3 \#4[a, b] of \cite {tw}
\end{problem}

\begin{problem}[15 points]
	\S 14.3 \#4[c, d] of \cite {tw}
\end{problem}

\begin{problem}[15 points]
	\S 14.3 \#5 of \cite {tw}
\end{problem}

\begin{problem}[15 points]
	\S 14.3 \#6 of \cite {tw}
\end{problem}

	\section{Identification Schemes and Authentication}
Read \S 14.2 of \cite{tw}.

	%aka Authentication
	\subsection{Message Authentication Codes}
	\label{sssec:mac}

\begin{definition}
A {\bf message authentication code (MAC)} is data transmitted with a message that verifies that the message came from the stated sender (authentication) and that the data has not been altered during the transmission (integrity).
\end{definition}

	The following are examples of MAC algorithms and frameworks
	\begin{itemize}
		\item CBC-MAC
		\item Checksum (using a hash function), e.g. md5sum, sha1sum
		\item HMAC
		\item Poly1305 and Poly1305-AES (Dan Bernstein)
	\end{itemize}


\begin{problem}  [15 points]

	\begin{enumerate}
		\item Who created the MAC, what is (are) their affiliation(s), and when was it developed?
		\item How is the MAC or MAC framework used?
		\item Discuss the details of the MAC or framework. Pictures help.
		\item Discuss the security of the MAC or MAC framework.
	\end{enumerate}
\end{problem}


	\section{Digital Signatures}
	%Also Blind signatures

	Read Chapter 9 of \cite{tw}.

	Another signature scheme is Rabin's Signature Algorithm. Suppose that Alice sends a message $M$ to Bob. She wants to sign the message to prove to Bob that she alone sent the message.

	\begin{enumerate}
		\item Alice chooses two distinct primes $p,q\equiv 3\ppmod{4}$ and keeps these two primes private. (Ideally, $p$ and $q$ are {\bf safe} primes, i.e. primes such that $p=2r+1$ and $q = 2s+1$, where $r$ and $s$ are prime.)
		\item Alice and Bob agree on a cryptographically secure hash function, $h$.
		\item Alice chooses a random padding $u$ and concatenates $m=M||u$.
		\item If $h(m)$ is not a quadratic residue modulo $p$ and modulo $q$, go back to Step 3.
		\item Alice computes $s = \sqrt{h(m)}$ and sends $s$ and $m$ to Bob.
		\item Bob verifies that $s^2 \equiv h(m) \ppmod{n}$.
	\end{enumerate}


\begin{problem}[10 points]
	Why is Rabin's Signature Algorithm secure?
\end{problem}

\begin{problem}[10 points]
	Choose two 3-digit safe primes and sign the message $M=1618$. Use a 1-digit pad and skip the use of a hash function.
\end{problem}

\begin{problem}[15 points]
	\S 9.6 \#1 of \cite {tw}
\end{problem}

\begin{problem}[15 points]
	\S 9.6 \#2 of \cite {tw}
\end{problem}

\begin{problem}[10 points]
	\S 9.6 \#3 of \cite {tw}
\end{problem}

\begin{problem}[10 points]
	\S 9.6 \#4[a] of \cite {tw}
\end{problem}

\begin{problem}[10 points]
	\S 9.6 \#4[b] of \cite {tw}
\end{problem}

\begin{problem}[10 points]
	\S 9.6 \#4[c] of \cite {tw}
\end{problem}

\begin{problem}[15 points]
	\S 9.6 \#5 of \cite {tw}
\end{problem}

\begin{problem}[10 points]
	\S 9.6 \#6 of \cite {tw}
\end{problem}

\begin{problem}[10 points]
	\S 9.6 \#7 of \cite {tw}
\end{problem}

\begin{problem}[15 points]
	\S 9.6 \#8 of \cite {tw}
\end{problem}

\begin{problem}[10 points]
	\S 9.7 \#1 of \cite {tw}
\end{problem}

\begin{problem}[10 points]
	\S 9.7 \#2 of \cite {tw}
\end{problem}

\begin{problem}[10 points]
	\S 9.7 \#3 of \cite {tw}
\end{problem}

	\section{Secret Sharing and Threshold Schemes}

Read Chapter 12 of \cite{tw}.
	%Activity: encrypt a password as the constant term of a linear congruence. Give each student a point on the line. Have them determine the password.
	% Have a "spy" with a fake point. Figure out who the spy is.

\begin{problem}[10 points]
	Let $p$ be the smallest prime greater than $2^{20}$ and let $q(x) = ax^2+bx+c \ppmod{p}$ be a quadratic polynomial modulo $p$. A PIN has been encoded as the constant term, $c$, of $q(x)$. Alice, Bob, and Charlie have been given the points of the curve $y=q(x)$: $(984544, 900317)$, $(248781, 118521)$, and $741200, 198787)$, respectively. Recover the PIN.
% 314531
\end{problem}

\begin{problem}[10 points]
	\S 12.3 \#1 of \cite {tw}
\end{problem}

\begin{problem}[10 points]
	\S 12.3 \#2 of \cite {tw}
\end{problem}

\begin{problem}[5 points]
	\S 12.3 \#3 of \cite {tw}
\end{problem}

\begin{problem}[10 points]
	\S 12.3 \#4 of \cite {tw}
\end{problem}

\begin{problem}[10 points]
	\S 12.3 \#5 of \cite {tw}
\end{problem}

\begin{problem}[15 points]
	\S 12.3 \#6 of \cite {tw}
\end{problem}

\begin{problem}[15 points]
	\S 12.3 \#7 of \cite {tw}
\end{problem}

\begin{problem}[10 points]
	\S 12.3 \#8 of \cite {tw}
\end{problem}

\begin{problem}[15 points]
	\S 12.3 \#9 of \cite {tw}
\end{problem}

\begin{problem}[15 points]
	\S 12.3 \#10 of \cite {tw}
\end{problem}

\begin{problem}[10 points]
	\S 12.3 \#11 of \cite {tw}
\end{problem}

\begin{problem}[10 points]
	\S 12.4 \#2 of \cite {tw}
\end{problem}

\begin{problem}[10 points]
	\S 12.3 \#3 of \cite {tw}
\end{problem}

	\section{Digital Coin-Flipping}

Read \S 13.1 of \cite{tw}.

\begin{problem}[10 points]
	\S 13.3 \#1[a] of \cite {tw}
\end{problem}

\begin{problem}[10 points]
	\S 13.3 \#1[b,c] of \cite {tw}
\end{problem}

\begin{problem}[10 points]
	\S 13.3 \#1[d] of \cite {tw}
\end{problem}

\begin{problem}[10 points]
	\S 13.3 \#2[a] of \cite {tw}
\end{problem}

\begin{problem}[15 points]
	\S 13.3 \#2[b,c] of \cite {tw}
\end{problem}

\begin{problem}[10 points]
	\S 13.3 \#2[d] of \cite {tw}
\end{problem}

\begin{problem}[15 points]
	\S 13.3 \#3[a, b] of \cite {tw}
\end{problem}

\begin{problem}[10 points]
	\S 13.3 \#3[c] of \cite {tw}
\end{problem}


	\section{Data Integrity}

Read \S 10.8 of \cite{tw}.

\begin{problem}[10 points]
	\S 10.9 \#4 of \cite {tw}
\end{problem}

\begin{problem}[10 points]
	\S 10.9 \#6 of \cite {tw}
\end{problem}

	%\section{Anonymity}

	%\section{The Socialist Millionaire Problem}
